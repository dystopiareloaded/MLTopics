\documentclass[12pt]{article}
\usepackage[a4paper, margin=1in]{geometry}
\usepackage{amsmath}
\usepackage{enumitem}
\usepackage{titlesec}
\titleformat{\section}{\large\bfseries}{\thesection}{1em}{}

\title{\textbf{University-Style Machine Learning Question Bank}\\\large (Topic-Wise: Weeks 23–36)}
\date{}

\begin{document}
\maketitle

\section*{Week 23: Introduction to ML \& Linear Regression}
\textbf{Theory}
\begin{itemize}
\item Define ML. Differentiate Supervised, Unsupervised, Semi-supervised, Reinforcement Learning.
\item Discuss challenges in ML system design.
\item Compare Batch vs. Online Learning.
\item Explain ML development life cycle.
\end{itemize}

\textbf{Derivation}
\begin{itemize}
\item Derive the Normal Equation for Simple Linear Regression.
\item Prove least squares minimizes sum of squared errors.
\end{itemize}

\textbf{Numerical}
\begin{itemize}
\item Compute slope and intercept manually.
\item Calculate MAE, MSE, RMSE, and $R^2$.
\end{itemize}

\section*{Week 24: Gradient Descent}
\textbf{Theory}
\begin{itemize}
\item Compare Batch, Stochastic, and Mini-Batch Gradient Descent.
\item Explain effect of learning rate.
\end{itemize}

\textbf{Derivation}
\begin{itemize}
\item Derive update rules for $\theta_0$ and $\theta_1$.
\item Show GD on convex vs. non-convex losses.
\end{itemize}

\textbf{Numerical}
\begin{itemize}
\item 3 iterations of GD with $\theta=0$.
\item Compare convergence using $\alpha = 0.01$ and $0.1$.
\end{itemize}

\section*{Week 25: Regression Analysis}
\textbf{Theory}
\begin{itemize}
\item Difference between inference and prediction.
\item Explain assumptions of Linear Regression.
\end{itemize}

\textbf{Derivation}
\begin{itemize}
\item Derive F-statistic using ESS, RSS, TSS.
\item Prove $R^2 = 1 - \frac{RSS}{TSS}$.
\end{itemize}

\textbf{Numerical}
\begin{itemize}
\item Compute $R^2$, Adjusted $R^2$, and F-statistic.
\item Interpret confidence intervals and p-values.
\end{itemize}

\section*{Week 26: Feature Selection}
\textbf{Theory}
\begin{itemize}
\item Compare Filter, Wrapper, and Embedded methods.
\item Limitations of correlation-based feature selection.
\end{itemize}

\textbf{Numerical}
\begin{itemize}
\item Calculate ANOVA F-value for a feature.
\item Compute VIF and interpret multicollinearity.
\end{itemize}

\section*{Week 27: Regularization}
\textbf{Theory}
\begin{itemize}
\item Explain bias-variance tradeoff with diagram.
\item Differences among Ridge, Lasso, and ElasticNet.
\end{itemize}

\textbf{Derivation}
\begin{itemize}
\item Derive Ridge Regression cost function and normal equation.
\item Show how L1 regularization leads to sparsity.
\end{itemize}

\textbf{Numerical}
\begin{itemize}
\item Compute Ridge coefficients for a dataset.
\item Plot coefficient paths for Lasso.
\end{itemize}

\section*{Week 28: K-Nearest Neighbors}
\textbf{Theory}
\begin{itemize}
\item Explain effect of $K$ on bias and variance.
\item Describe distance metrics used in KNN.
\end{itemize}

\textbf{Numerical}
\begin{itemize}
\item Classify a point using $K=3$ on given data.
\item Perform weighted KNN prediction.
\end{itemize}

\section*{Week 29: Principal Component Analysis (PCA)}
\textbf{Theory}
\begin{itemize}
\item What is the curse of dimensionality? How does PCA help?
\item Compare PCA and SVD.
\end{itemize}

\textbf{Derivation}
\begin{itemize}
\item Derive PCA using variance maximization.
\item Explain role of eigenvalues and eigenvectors.
\end{itemize}

\textbf{Numerical}
\begin{itemize}
\item Standardize a 2D dataset, compute covariance matrix.
\item Perform eigen-decomposition and project onto principal component.
\end{itemize}

\section*{Week 30: Model Evaluation and Selection}
\textbf{Theory}
\begin{itemize}
\item Explain ROC-AUC curve and confusion matrix.
\item Discuss precision-recall trade-off.
\end{itemize}

\textbf{Numerical}
\begin{itemize}
\item Calculate accuracy, precision, recall, F1-score from confusion matrix.
\item Compute k-fold cross-validation score.
\end{itemize}

\section*{Week 31: Naive Bayes}
\textbf{Theory}
\begin{itemize}
\item State Naive Bayes assumption.
\item Compare Gaussian, Multinomial, and Bernoulli Naive Bayes.
\end{itemize}

\textbf{Derivation}
\begin{itemize}
\item Derive Bayes' Theorem and its application to classification.
\item Explain need for log probabilities and Laplace smoothing.
\end{itemize}

\textbf{Numerical}
\begin{itemize}
\item Calculate posterior class probability for a text.
\item Build spam classifier using word frequencies.
\end{itemize}

\section*{Week 32: Logistic Regression}
\textbf{Theory}
\begin{itemize}
\item Explain sigmoid function and its interpretation.
\item Define odds, log-odds and their use in interpretation.
\end{itemize}

\textbf{Derivation}
\begin{itemize}
\item Derive log-likelihood function for logistic regression.
\item Show gradient descent update for logistic loss.
\end{itemize}

\textbf{Numerical}
\begin{itemize}
\item Perform 1 iteration of gradient descent.
\item Calculate log loss for a small prediction set.
\end{itemize}

\section*{Week 33: Support Vector Machines}
\textbf{Theory}
\begin{itemize}
\item Compare hard and soft margin SVM.
\item Explain purpose of kernel functions.
\end{itemize}

\textbf{Derivation}
\begin{itemize}
\item Derive SVM optimization formulation.
\item Explain KKT conditions in context of SVM.
\end{itemize}

\textbf{Numerical}
\begin{itemize}
\item Identify support vectors from a plot.
\item Compute RBF kernel value between two points.
\end{itemize}

\section*{Week 34: Decision Trees}
\textbf{Theory}
\begin{itemize}
\item Explain CART algorithm using Gini impurity.
\item What is pruning? Why is it needed?
\end{itemize}

\textbf{Numerical}
\begin{itemize}
\item Compute Gini index and Information Gain.
\item Build a decision tree of depth 2 manually.
\end{itemize}

\section*{Week 35: Ensemble Methods}
\textbf{Theory}
\begin{itemize}
\item Compare Bagging and Boosting.
\item Explain working of Random Forest.
\end{itemize}

\textbf{Numerical}
\begin{itemize}
\item Simulate 3-tree Random Forest manually.
\item Perform 2 rounds of AdaBoost.
\end{itemize}

\section*{Week 36: Gradient Boosting and XGBoost}
\textbf{Theory}
\begin{itemize}
\item Explain working of Gradient Boosting.
\item How does XGBoost differ from traditional GBM?
\end{itemize}

\textbf{Derivation}
\begin{itemize}
\item Derive function update for log-loss classification.
\item Explain regularization using Taylor expansion.
\end{itemize}

\textbf{Numerical}
\begin{itemize}
\item Perform one iteration of Gradient Boosting.
\item Compute similarity score and gain in XGBoost.
\end{itemize}

\section*{Clustering Algorithms (KMeans, DBSCAN, Hierarchical)}
\textbf{Theory}
\begin{itemize}
\item Compare Partitional, Hierarchical, and Density-based clustering.
\item Explain Elbow method and Silhouette Score.
\end{itemize}

\textbf{Derivation}
\begin{itemize}
\item Derive update rule for Lloyd’s algorithm in KMeans.
\end{itemize}

\textbf{Numerical}
\begin{itemize}
\item Perform 2 iterations of KMeans on 2D data.
\item Classify points as core/border in DBSCAN.
\item Construct dendrogram and determine number of clusters.
\end{itemize}

\end{document}
