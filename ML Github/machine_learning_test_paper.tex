\documentclass[12pt]{article}
\usepackage[a4paper, margin=1in]{geometry}
\usepackage{amsmath}
\usepackage{graphicx}
\usepackage{enumitem}
\usepackage{fancyhdr}
\pagestyle{fancy}
\fancyhf{}
\rhead{Semester Exam – Machine Learning}
\lhead{University of Data Sciences}
\rfoot{Page \thepage}

\title{\textbf{University of Data Sciences} \\ \vspace{0.5em} \Large Semester Examination – Machine Learning}
\date{}
\begin{document}

\maketitle
\vspace{-1em}
\noindent\textbf{Duration:} 3 Hours \hfill \textbf{Full Marks:} 100\\
\textbf{Instructions:}
\begin{itemize}
  \item Attempt all questions from \textbf{Section A (Compulsory)}
  \item Answer \textbf{any 3} questions from \textbf{Section B} and \textbf{any 2} from \textbf{Section C}
  \item Show all necessary steps in derivations and numericals
  \item Use appropriate diagrams/visuals where applicable
\end{itemize}

\section*{Section A: Compulsory – 30 Marks (Short Answer)}
\begin{enumerate}[label=\arabic*.]
    \item Define Machine Learning. Differentiate between Supervised and Unsupervised learning.
    \item What is the role of the learning rate in gradient descent?
    \item Mention two assumptions of Linear Regression.
    \item Define $R^2$ and Adjusted $R^2$. Why is Adjusted $R^2$ preferred in multiple regression?
    \item Write the cost function for Ridge Regression.
    \item What is the curse of dimensionality in the context of KNN?
    \item List any two key differences between Logistic Regression and Linear Regression.
    \item Define VIF and its use in detecting multicollinearity.
    \item What is data leakage? Give one example.
    \item Compare Batch, Stochastic, and Mini-batch Gradient Descent.
\end{enumerate}

\section*{Section B: Theory \& Derivation – 40 Marks (Any 3 $\times$ 13.33)}

\textbf{Q1. Linear Regression \& Error Metrics}
\begin{itemize}
  \item[(a)] Derive the normal equation for Simple Linear Regression.
  \item[(b)] Define and derive the formula for Mean Squared Error (MSE).
  \item[(c)] Explain the geometrical interpretation of regression line fitting.
\end{itemize}

\textbf{Q2. Gradient Descent}
\begin{itemize}
  \item[(a)] Derive the update rule for $\theta_0$ and $\theta_1$ in gradient descent for linear regression.
  \item[(b)] Explain how learning rate affects convergence.
  \item[(c)] What happens when the cost function is non-convex?
\end{itemize}

\textbf{Q3. Regularization Techniques}
\begin{itemize}
  \item[(a)] Explain the bias-variance tradeoff with a diagram.
  \item[(b)] Derive the Ridge Regression cost function and show how it modifies the normal equation.
  \item[(c)] Contrast Lasso and Ridge in terms of feature selection.
\end{itemize}

\textbf{Q4. Principal Component Analysis (PCA)}
\begin{itemize}
  \item[(a)] Derive PCA from first principles using variance maximization.
  \item[(b)] Explain the role of eigenvectors and eigenvalues in PCA.
  \item[(c)] Why is standardization important before applying PCA?
\end{itemize}

\textbf{Q5. Logistic Regression}
\begin{itemize}
  \item[(a)] Derive the log-likelihood function for logistic regression.
  \item[(b)] Explain how gradient descent is applied to logistic regression.
  \item[(c)] Interpret the meaning of coefficients in logistic regression.
\end{itemize}

\section*{Section C: Numericals \& Applied Problems – 30 Marks (Any 2 $\times$ 15)}

\textbf{Q6. Multiple Linear Regression \& Multicollinearity}
\begin{itemize}
  \item Given a dataset with 3 predictors and a dependent variable, perform multiple linear regression and compute $R^2$ and Adjusted $R^2$.
  \item Also, calculate VIF for each predictor and interpret.
\end{itemize}

\textbf{Q7. Naive Bayes Classifier (Numerical)}
\begin{itemize}
  \item You are given data about email spam classification. Using Naive Bayes, calculate the probability that a new email with specific word counts is spam.
  \item Include Laplace smoothing.
\end{itemize}

\textbf{Q8. PCA Numerical}
\begin{itemize}
  \item Given the 2D dataset:
  \[
  X = \begin{bmatrix}
  2 & 0 \\
  0 & 2 \\
  1 & 1
  \end{bmatrix}
  \]
  \item (a) Standardize the data  
  \item (b) Compute the covariance matrix  
  \item (c) Extract eigenvalues and eigenvectors  
  \item (d) Project the data to 1D using PCA
\end{itemize}

\textbf{Q9. Gradient Descent Simulation}
\begin{itemize}
  \item Implement gradient descent from scratch (pseudo-code allowed) for the cost function:
  \[
  J(\theta) = \frac{1}{2m} \sum_{i=1}^{m} (h_\theta(x^{(i)}) - y^{(i)})^2
  \]
  \item Use initial $\theta = 0$, learning rate = 0.01, and 3 data points. Show 3 iterations manually.
\end{itemize}

\textbf{Q10. Classification Metrics \& Confusion Matrix}
\begin{itemize}
  \item A binary classifier gives the following results: TP = 50, FP = 10, TN = 30, FN = 10.
  \item Calculate: Accuracy, Precision, Recall, F1-score
  \item Draw the confusion matrix.
\end{itemize}

\section*{Bonus Question (5 Marks)}
\textbf{Q11.} What are the advantages and limitations of using Ensemble Methods like Random Forest and Gradient Boosting in real-world datasets?

\end{document}
